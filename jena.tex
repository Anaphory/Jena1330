\documentclass{article}

\usepackage[utf8]{inputenc}

%
% Jena1330 inline font declaration
%

\usepackage[T1]{fontenc}

\DeclareFontFamily{T1}{jena}{}
\DeclareFontShape{T1}{jena}{m}{n}{<-> Jena1330}{}

\DeclareRobustCommand\jena{\fontfamily{jena}\selectfont}
\DeclareTextFontCommand{\textjena}{\jena}

% The following line assumes that this is compiled with pdflatex
% and that the files Jena1330.tfm and Jena1330.pfb are available
% in the search path.

\pdfmapline{+Jena1330 <Jena1330.pfb}

% end font declaration

\begin{document}

\noindent
The following text demonstrates \textjena{Jena1330}:
\bigskip

{\Large\jena \def\TeX{TeX}%
 \noindent\input II3 }
\bigskip

\noindent
Here are the ligatures:\par
\textjena{fl fi}

\noindent
These are the characters with init, medi, or final forms:\par
\textjena{fifif gigig sisis viviv wiwiw titit}

\noindent
And then there is the r~\ldots\par
\textjena{Brx Drx Grx Orx Prx Srx Urx Vrx Wrx Yrx brx drx orx prx yrx}

\noindent
I'm not sure if this is intentional:\par
\textjena{vrx} vs. \textjena{xvrx}\par
\textjena{wrx} vs. \textjena{xwrx}\par

\noindent
Same thing here (or for other boundary characters):\par
\textjena{vxf} vs. \textjena{(vxf)}

\end{document}
